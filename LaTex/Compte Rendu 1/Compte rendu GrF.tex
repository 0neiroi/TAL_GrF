\documentclass[12pt]{report}

\usepackage[latin1]{inputenc}
\usepackage[T1]{fontenc}
\usepackage[francais]{babel}

\title{Compte Rendu de l'avanc�e du projet}
\author{Projet de TAL, Groupe F}
\date{17/03/2017}
\begin{document}

%Page de garde
\maketitle

%Compositionn de l'�quipe
\section*{Composition de l'�quipe}
Equipe F :
\begin{itemize}
\item BIVER Alexis
\item BRICHE Arnaud
\item COLIN Emile
\item CULARD Mathieu
\item RICHIER Val�re
\end{itemize}

%Avanc�e du projet
\section*{Avanc�e du projet}
\begin{itemize}
\item Mise en place d'un espace de travail collaboratif :
\begin{itemize}
	\item Trello
	\item GitHub
	\item D�finition des t�ches et d�but de r�partition
	\end{itemize}
\item Recherche de mots cl�s pour la composition du corpus
\item Extraction du corpus
\item R�flexions sur l'impl�mentation de programmes en Java
\item Acquisitions de comp�tences en LaTex via OpenClassroom
\end{itemize}

%Choix des mots cl�s
\section*{Choix des mots cl�s}
Mots cl�s choisis :
\begin{itemize}
\item fictional assassin
\item murderers
\end{itemize}
"Fictional assassin" nous permet de nous d�marquer des autres groupes en cherchant aussi bien des assassins de fiction que des assassins r�els, augmentant la diversit� de nos r�sultats.


\end{document}
